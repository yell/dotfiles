\documentclass[a4paper,11pt]{article}

\usepackage[symbol]{footmisc}  % footnote symbols instead of numbers
\usepackage{movie15}  % inline figures
\usepackage[utf8]{inputenc} % allow utf-8 input
\usepackage[T1]{fontenc}    % use 8-bit T1 fonts

% --- microtype ---
\usepackage[activate={true,nocompatibility},
            factor=1100,
            stretch=10,
            shrink=10]{microtype}  % improve general appearance
% activate - protrusion and expansion
% factor - add 10% to the protrusion amount (default is 1000)
% stretch / shrink - reduce stretchability/shrinkability (default is 20/20)

% \usepackage{palatino}  % ugly font [deprecated]
% \usepackage[sc]{mathpazo}  % same font but not deprecated package
\usepackage{pifont}  % special PostScript fonts (e.g. for xmark == ding{53})
% \usepackage[scaled=0.8]{luximono}  % fixed-width font that supports boldface (useful when typesetting source code)

% \usepackage{multicol}  % typeset text in multiple columns
\usepackage[titletoc]{appendix}  % prepend "Appendix" to chapter names
\usepackage{titlesec}  % help with section naming
% \usepackage{geometry}  % page dimensions, margins ...
% \usepackage{fancyhdr}  % customizing the headers and footers
% \usepackage{quotchap}  % fancy chapter heading pages
% \usepackage[parfill]{parskip}  % use no indentation and space between paragraphs
\usepackage{nicefrac}  % nicer fractions


% ----------------------------
% ----- Utility packages -----
\usepackage{empheq}  % for colorbox
\usepackage{mdframed}  % for box around figures

\usepackage{lipsum}
\usepackage{blindtext}

\usepackage{enumitem}  % much easier to make modifications to lists
% \usepackage{paralist}  % Improves enumerate and itemize. Also provides some compact environments

\usepackage{ifthen}  % used in \clearemptydoublepage
\usepackage{rotating}  % rotation tools, including rotated full-page floats
\usepackage{xparse}  % more flexible macros with more than one optional argument

% --- margin notes ---
\usepackage{marginnote}
\usepackage{mparhack}
\usepackage{marginfix}


% ------------------------------
% ----- Document structure -----
\usepackage{booktabs}  % for elegant tables
\usepackage[acronym,symbols,
            nomain,nogroupskip,nonumberlist,nopostdot,  % xindy,
            toc]{glossaries}
\usepackage{glossary-longbooktabs}
  \newglossary[slg]{symbolslist}{syi}{syg}{Symbolslist}
  \makeglossaries
% \usepackage{makeidx}  % standard LaTeX package for creating indexes
% xindy - more flexible than makeindex (sudo apt install xindy)
\usepackage{subfiles}


% ------------------------
% ----- Bibliography -----
\usepackage{csquotes}
\usepackage[english]{babel}  % set default language to English
% \usepackage[backend=biber,
%             bibstyle=authoryear,
%             citestyle=authoryear,%apa,
%             % natbib=true,
%             hyperref=true,
%             isbn=true,
%             doi=true,
%             url=true,
%             eprint=true]{biblatex}
% \DeclareLanguageMapping{english}{english-apa}

% --- apacite ---
\usepackage[natbibapa]{apacite}
  \bibliographystyle{apacite}
\usepackage[hyphens,spaces]{url}

% --- plain natbib ---
% \usepackage[round,sort]{natbib}
% \bibliographystyle{plainnat}

\usepackage[nottoc,numbib]{tocbibind}  % add bibliography & listings to ToC

% ------------------
% ----- Tables -----
\usepackage{tabularx}
\usepackage{tabu}  % tables
% \usepackage{colortbl}  % add color to LaTeX tables
% \usepackage{multirow}  % tabular cells spanning multiple rows
% \usepackage{dcolumn}  % columns of different types and alignment


% ----------------
% ----- Math -----

% --- AMS facilities ---
\usepackage{amsmath}
\usepackage{amsthm}
\usepackage{amsfonts}
\usepackage{amssymb}

\usepackage{mathtools}
\usepackage[retainorgcmds]{IEEEtrantools}  % sophisticated equation arrays
% \usepackage{yhmath}  % extended maths fonts for LaTeX
% \usepackage{commath}  % math delimiters of auto computed size
% \usepackage{comment}
% \usepackage{kbordermatrix}  % add column and row labels to matrices
\usepackage{physics}  % \bra, \ket, \expval, \mel ...
                      % & \abs, \norm, \var, \tr, \Tr ...
\usepackage{bbm}
\usepackage[makeroom]{cancel}


% -------------------------------
% ----- Graphics and floats -----
\usepackage{float}  % improved interface for floating objects
\usepackage{caption}
\usepackage{subcaption}  % to use subfigures
\usepackage{graphicx}  % enhanced support for graphics
  \graphicspath{{figures/}{../figures/}}  % in gereral {{subdir1/}...{subdirn/}}

% \usepackage{tikz}
% \usepackage{pgfplots}  % uses PGF to draw professionally looking charts and plots

% -----------------
% ----- Color -----
\usepackage[dvipsnames]{xcolor}  % color (+ in math mode)
% pre-defined colors: https://en.wikibooks.org/wiki/LaTeX/Colors#Predefined_colors

% --- my colors ---
\definecolor{tumblue}{HTML}{0065BD}
\definecolor{darkblue}{HTML}{001473}
\definecolor{darkgray}{RGB}{100, 100, 100}
\definecolor{ddarkgray}{RGB}{66, 66, 66}
\definecolor{gray}{rgb}{0.5,0.5,0.5}
\definecolor{orange}{rgb}{1.0,0.3,0.01}
\definecolor{violet}{rgb}{0.4,0.0,0.6}
\definecolor{yellow}{rgb}{1.0,0.7,0.0}
\definecolor{boxc}{rgb}{1.0, 1.0, .9}
\definecolor{mygreen}{rgb}{0, 0.75, 0.18}

% --- hyperref colors ---
\def\linkcolor{tumblue}
\def\urlcolor{darkblue}
\def\citecolor{darkblue}


% -----------------------
% ----- Source code -----
% \usepackage{listings} % (optional - alternative)
% \usepackage[newfloat]{minted} % (recommended)
% Set global Minted options
% \setminted{linenos, autogobble, frame=lines, framesep=2mm}
%%% Inline C++ (optional)
% \newcommand{\incpp}[1]{\mintinline{c++}{#1}}
% \newenvironment{code}{\captionsetup{type=listing}}{}
% \SetupFloatingEnvironment{listing}{name=Source Code}
% --------------
% python listing
% --------------
% \floatstyle{plain}
% \restylefloat{figure}
% \lstset{frame=tb,
% 	language=Python,
% 	aboveskip=3mm,
% 	belowskip=3mm,
% 	showstringspaces=false,
% 	columns=flexible,
% 	basicstyle={\small\ttfamily},
% 	numbers=none,
% 	numberstyle=\color{violet},
% 	keywordstyle=\color{orange},
% 	commentstyle=\color{gray},
% 	stringstyle=\color{yellow},
% 	breaklines=false,
% 	breakatwhitespace=false,
% 	tabsize=4
% }


% -----------------------------------
% ----- Algorithms (pseudocode) -----
\usepackage{algorithmicx}
\usepackage{algpseudocode}
  \algrenewcommand{\algorithmiccomment}[1]{\hfill$\rightarrow$ #1}  % normal arrow comments


% --------------------
% ----- Hyperref -----
\usepackage[linktocpage,
            colorlinks=true,
            bookmarksnumbered=true]{hyperref}

% set links colors
\hypersetup{
  linkcolor=\linkcolor,%
  urlcolor=\urlcolor,%
  citecolor=\citecolor
}

% disable the coloring of the links when printing
\usepackage[ocgcolorlinks]{ocgx2}[2017/03/30]

% PDF metadata
% \hypersetup{
%   pdftitle={\metatitle},%
%   pdfauthor={\metaauthor},%
%   pdfkeywords={\metakeywords},%
%   pdfsubject={\metasubject}
% }

% ----------------------------------------------------------------------------------
% ----- Utilities that must go after other packages (`hyperref`, `xcolor` ...) -----
\usepackage{bookmark}  % better bookmark handling (loads hyperref)
\usepackage{hyperxmp}  % create XMP Metadata (uses the values from hyperref)
\usepackage{xspace}  % define commands that do not eat spaces
% \usepackage{thumbpdf}  % make thumbnails
\usepackage{todonotes}
% \usepackage{fancyvrb}  % highly customisable verbatim

% # General LaTeX shortcuts
\renewcommand{\u}[1]{\underline{#1}}


% # Custom symbols
\newcommand{\gooditem}{\item[\checkmark]}
\newcommand{\baditem}{\item[\ding{53}]}
\newcommand{\good}{\textbf{\color{green}[\checkmark]}}
\newcommand{\bad}{\textbf{\color{red}[\ding{53}]}}


% # Document formatting
% ## page clearing
\newcommand{\clearemptydoublepage}{%
  \ifthenelse{\boolean{@twoside}}{\newpage{\pagestyle{empty}\cleardoublepage}}%
  {\clearpage}}

% ## comment that appears on the border
\newcommand{\mcomment}[1]{\marginpar{\raggedright \noindent {\textsl{#1}}}}


% # Graphics
% ## insert figure (fname, caption for ToC, caption, label, width as fraction of \textwidth)
\newcommand{\insertfig}[5]{
	\begin{figure}[htbp]
		\begin{center}
			\includegraphics[width=#5\textwidth]{#1}  % width, height, scale, angle ...
		\end{center}
		\vspace{-0.4cm}
		\caption[#2]{#3}
		\label{fig:#4}
	\end{figure}
}

% ## color box
\newlength\mytemplen
\newsavebox\mytempbox

\makeatletter

\newcommand\mybox{%
    \@ifnextchar[%]
       {\@mybox}%
       {\@mybox[0pt]}}

\def\@mybox[#1]{%
    \@ifnextchar[%]
       {\@@mybox[#1]}%
       {\@@mybox[#1][0pt]}}

\def\@@mybox[#1][#2]#3{
    \sbox\mytempbox{#3}%
    \mytemplen\ht\mytempbox
    \advance\mytemplen #1\relax
    \ht\mytempbox\mytemplen
    \mytemplen\dp\mytempbox
    \advance\mytemplen #2\relax
    \dp\mytempbox\mytemplen
    \colorbox{boxc}{\hspace{0em}\usebox{\mytempbox}\hspace{0em}}}

\makeatother


% ## title
\title{Title}
\author{
  Author\thanks{thanks}\\
  \texttt{email1}
  \and
  Author2\thanks{thanks}\\
  \texttt{email2}
}
\date{}

% ## bibliography using `biblatex`
% \addbibresource{bibliography/bib.bib}

\begin{document}
  \maketitle


  \section{Essential} \label{sec:label}
  % ## common
  % \tableofcontents
  % \listoftables
  % \listoffigures
  % {\let\cleardoublepage\clearpage \chapter{Chapter}\label{ch:label}}

  % ## annotation
  % \listoftodos
  \todo{\textbackslash todo}
  \marginpar{\textbackslash marginpar}
  \mcomment{\textbackslash mcomment}

  % ## equation
  \begin{IEEEeqnarray}{rCl}
    \binom{n}{k}
      &=& \frac{n!}{k!(n-k)!} \label{eq:label1}
      \\[0.5em]
      &=& \frac{1}{2\pi i}\oint\limits_{\Gamma} \frac{ (1+z)^n }{z^{k+1}} \ud z
      \label{eq:label2}
  \end{IEEEeqnarray}

  % ## table
  \begin{table}[h]
    \caption{Caption}
    \label{tab:label}
    \centering
    \begin{tabular}{cc}
      \toprule
      A & B\footnotemark \\
      \midrule
      a & b \\
      c & d \\
      \midrule
      e & f \\
      g & h \\
      \bottomrule
    \end{tabular}
  \end{table}

  \footnotetext{footnotemark--footnotetext}

  % ## figure
  \insertfig{death-star}{Caption for ToC}{Caption}{label}{0.2}  % `fig:label`

  % ## cite
  \noindent TensorFlow\footnote{footnote} \citep{tf}, \citet{tf}.\\
  Section \ref{sec:label} on a page \pageref{sec:label},
  table \ref{tab:label}, figure \ref{fig:label},
  equations \eqref{eq:label1} and \eqref{eq:label2}.


  \section{Other CO\texorpdfstring{\textsubscript{2}}{2}}

  % ## subfigures
  % \begin{figure}[h]
  %   \centering
  %   \begin{subfigure}[b]{0.5\textwidth}
  %     \includegraphics[width=\textwidth]{death-star}
  %     \caption{Caption 1}
  %     \label{fig:1}
  %   \end{subfigure}
  %   %
  %   \begin{subfigure}[b]{0.5\textwidth}
  %     \includegraphics[width=\textwidth]{death-star}
  %     \caption{Caption 2}
  %     \label{fig:2}
  %   \end{subfigure}
  %   %
  %   \begin{subfigure}[b]{0.33\textwidth}
  %     \includegraphics[width=\textwidth]{death-star}
  %     \caption{Caption 3}
  %     \label{fig:3}
  %   \end{subfigure}
  %   %
  %   \begin{subfigure}[b]{0.33\textwidth}
  %     \includegraphics[width=\textwidth]{death-star}
  %     \caption{Caption 4}
  %     \label{fig:4}
  %   \end{subfigure}
  %   %
  %   \begin{subfigure}[b]{0.33\textwidth}
  %     \includegraphics[width=\textwidth]{death-star}
  %     \caption{Caption 5}
  %     \label{fig:5}
  %   \end{subfigure}
  % \end{figure}

  \subsection*{Proof}
    \begin{proof}[\unskip\nopunct]
      The proof is easy and is left to a reader.
    \end{proof}

  \subsection*{Test math}
    \begin{gather*}
      \bra{\frac{\Psi}{1}} \; \ket{\frac{\Psi}{1}} \;
      \braket{\frac{\Psi}{1}} \;
      \matrixelement{n}{\prod_k U_k}{\frac{x}{1}} \;
      \mel{n}{\prod_k U_k}{\frac{x}{1}} \\
      \text{Normal}(\mathrm{x} \mid \mu, \sigma^2 ) \\
      \textnormal{Normal}(\mathrm{x} \mid \mu, \sigma^2 ) \\
      \mathrm{Normal}(\mathrm{x} \mid \mu, \sigma^2 ) \\
      \mathcal{N}(\mathrm{x} \mid \mu, \sigma^2 ) \\
      \sum_{\mathclap{n=-\infty}}^{+\infty} f(x) \geqslant \geq \ge \med X \\
      \eps + \e^{-\frac{(x-2)^2}{2\sigma ^2}} + \const{} \\
      \dot{a} \epsilon \phi \varphi \\
      \not\propto \not\subset \not\in \notin \\
      \equiv \doteq \approx \subset \supset \ni \mid \parallel \neq \ne \\
      \Tr A = \tr A = \var X = \KL{P}{Q} = \DKL{P}{Q} \\
      \star \ast \circ \bullet \oplus \otimes \odot \dagger \ddagger \text{\dag \ddag} \\
      \bigoplus \bigotimes \bigodot \bigcup \bigcap \\
      \leftarrow \gets \rightarrow \to \mapsto
      \Leftarrow \Rightarrow \Longleftrightarrow \iff \overrightarrow{AB} \rightrightarrows \\
      []\lbrack \rbrack \lbrace \rbrace \langle \rangle
      | \vert \| \Vert \lfloor \rfloor \arrowvert \Arrowvert \\
      \ell \emptyset \Re \Im \bot \top \angle \Box \\
      \thicksim \thickapprox \backsim \varpropto \risingdotseq \fallingdotseq \\
      \hbar \square \blacksquare \bigstar \varnothing
    \end{gather*}

    \begin{equation}
      \begin{Vmatrix}
        1 & 2 \\
        3 & 4
      \end{Vmatrix}
      =
      \abs{\oint_{A}^{B} f(z) \ud z}
      =
      \frac{\d u}{\d x}
      = \mathcal{F}\mathfrak{F}
      = \frac{\displaystyle \sum a_{ij} }{\displaystyle \sum b_{ij} }
      = \sum_{\mathclap{\text{big long thing}}} a_k
    \end{equation}

    \begin{equation}
      = \P{\frac{X}{\E X} \leq \epsilon}
      = \Pr{\mathrm{Poisson}(\lambda=3) > 5}
      = \pd{x} \cdot \pd[f]{x} \cdot \pd[f][2]{x}
    \end{equation}

    \begin{equation}
      \o{a} \enskip
      A \stackrel{*}{\approx} B \enskip
      \sum_{\substack{0<i<n \\ j \neq i}}f(i) \enskip
      \sqrt[3]{P(x)+Q(x)} \enskip
      \frac{3}{8} \tfrac{3}{8} \dfrac{3}{8} 3/8 \enskip
      x=x\phantom{=}x\mathbin{=}x
    \end{equation}

  \subsection*{Math fonts}
    \begin{gather}
      \mathrm{ABCDEFabcdef} \tag{roman} \\
      \mathbf{ABCDEFabcdef} \tag{boldface} \\
      \mathsf{ABCDEFabcdef} \tag{sans serif} \\
      \mathtt{ABCDEFabcdef} \tag{typewriter} \\
      \mathit{ABCDEFabcdef} \tag{italic} \\
      \mathcal{ABCDEFabcdef} \tag{calligraphic} \\
      \mathnormal{ABCDEFabcdef} \tag{normal}
      \\[0.5em]
      \boldsymbol{ABCabc \Gamma\Omega\Xi \gamma\omega\xi } \tag{boldsymbol} \\
      \mathscr{ABCDEFabcdef} \tag{scr} \\
      \mathfrak{ABCDEFabcdef} \tag{frak} \\
      \mathbb{ABCDEFabcdef12345} \tag{bb} \\
      \mathbbm{ABCDEFabcdef12345} \tag{bbm}
    \end{gather}

  \subsection*{Text fonts}
    \noindent
    \textrm{ABCDEFabcdef}  % RoMan
    \textsf{ABCDEFabcdef}  % Sans seriF
    \texttt{ABCDEFabcdef}  % Type wriTer
    \\[0.5em]
    \textmd{ABCDEFabcdef}  % MeDium
    \textbf{ABCDEFabcdef}  % Bold Face
    \\[0.5em]
    \textup{ABCDEFabcdef}  % UPright
    \textit{ABCDEFabcdef}  % ITalic
    \textsl{ABCDEFabcdef}  % SLanted
    \textsc{ABCDEFabcdef}  % Small Caps
    \\[0.5em]
    \textnormal{ABCDEFabcdef}  % document font
    \emph{ABCDEFabcdef}    % EMPHasize

  \subsection*{General formatting}
    \begin{itemize}[noitemsep]
      \item x \hspace{\stretch{1}} y \hspace{\stretch{1}} z \hspace*{\stretch{1}}
      \item ``quote''
      \item Ph.~D.
      \item Ph.\ D.
      \item Ph. D.
      \item A. B
      \item A\@. B
      \item \verb*+yo wazup+
    \end{itemize}

  % # Bibliography
  % ## bibliography using `biblatex`
  % \setlength\bibitemsep{1.5\itemsep}
  % \printbibliography

  % ## bibliography using `natbib`
  \setlength\bibsep{1.5\itemsep}
  \renewcommand{\bibname}{Bibliography}
  \bibliography{bibliography/bib}

\end{document}
